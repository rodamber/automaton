\documentclass[12pt, a4paper]{article}

\usepackage{hyperref}
\usepackage{xspace}

\usepackage[T1]{fontenc}
\usepackage[utf8]{inputenc}
\usepackage{amsfonts}
\usepackage{amsmath}
\usepackage{amssymb}
\usepackage{amsthm}
\usepackage{calrsfs}

\DeclareMathAlphabet{\pazocal}{OMS}{zplm}{m}{n}

\newtheoremstyle{break}% name
  {}%         Space above, empty = `usual value'
  {}%         Space below
  {\itshape}% Body font
  {}%         Indent amount (empty = no indent, \parindent = para indent)
  {\bfseries}% Thm head font
  {.}%        Punctuation after thm head
  {\newline}% Space after thm head: \newline = linebreak
  {}%         Thm head spec
\theoremstyle{break}

\newtheorem{definition}{Definition}

\newcommand{\email}[1]{\href{mailto:#1@epfl.ch}{#1@epfl.ch}}

\newcommand{\nfa}{nondeterministic finite automaton\xspace}
\newcommand{\NFA}{Nondeterministic Finite Automaton\xspace}
\newcommand{\dfa}{deterministic finite automaton\xspace}
\newcommand{\DFA}{Deterministic Finite Automaton\xspace}

\begin{document}
\title{Semester Project Report}

\author{
  Rodrigo Bernardo\\
  \email{rodrigo.moreirabernardo}\\
}

\date{{January 2018}}
\maketitle

\section{Introduction}

The problem solved in this semester project was the development of a provably
correct \nfa (NFA) to \dfa (DFA) conversion.

The motivation for the DFA to NFA transformation came as part of a project
called \textit{lexi}. \textit{lexi} is a simple implementation of a lexer
generator, written in Scala. It provides a small DSL that allows you to define
tokens using regular expressions and compiles them to a DFA. The DFA outputs a
sequence of tokens when given an input string.

The DFA is obtained in two steps: we start by compiling the regular expression
to an NFA and then transform it to a DFA. In this project, we focus on the
latter step.

Much of the difficulty came from the fact that the implementation for sets in
\textit{Stainless} is limited. For example, while it is possible to check
element membership, there is no way to actually traverse the elements of the
underlying structure. Also, it is not possible to check the cardinality of the
set, as that functionality is broken (\textit{Stainless} does not compile any
program with calls to the \textit{size} method).

With this it became obvious that a solution for this problem would need a new
set implementation. However, with that, properties about sets that we usually
take for granted, such as, e.g., reflexivity and transitivity of equality, or
properties about the relative size of a set and one of its subsets, all need to
be proven. Properties that otherwise \textit{Stainless} would be able to
easily verify, had we used the set implementation in the \textit{Stainless}
library.

In the end, the proof is complete apart from two quirks that we believe are
shortcomings of \textit{Stainless}.

% *TODO*: Difficulties with understanding \textit{Stainless} and its shortcomings.
 % Documentation... 

% *TODO*: Main difficulty was also figuring out a good representation for automata
 % and sets that made the proofs easier

% *TODO*: \textit{Stainless} version used: master branch, commit ddff4c9

% \$ stainless --timeout=20 --verification --termination --watch --vc-cache USet.scala Set.scala Automaton.scala AutomatonCache.scala

\section{Background Theory}

We start by stating our definitions of DFA and NFA, as well as the conversion
algorithm.

\begin{definition}
  A deterministic finite automaton (DFA) is a quintuple
$(Q,\Sigma,\delta,q_0,F)$, where:
\begin{itemize}
  \item $Q$ is a finite set (the states);
  \item $\Sigma$ is a finite set (the alphabet);
  \item $\delta$ is a function defined over $Q x \Sigma$ with values in $Q$ (the transition
     function);
  \item $q_0$ is an element of $Q$ (the initial state);
  \item and $F$ is a subset of $Q$ (the final states).
\end{itemize}
\end{definition}

\begin{definition}
    We say that the DFA $A = (Q,\Sigma,\delta,q_0,F)$ accepts the word $w = w_1
\ldots w_n$, $w_i \in \Sigma$, $ 1 \leq i \leq n$, if there is a sequence of
states $r_0, \ldots, r_n$, $r_i \in Q$, $0 \leq i \leq n$, such that:
\begin{itemize}
  \item $r_0 = q_0$;
  \item $r_{i+1} = \delta(r_i, w_{i+1})$, $0 \leq i \leq n-1$;
  \item and $r_n \in F$.
\end{itemize}
\end{definition}

\begin{definition}
  The language recognized by the automaton $A = (Q,\Sigma,\delta,q_0,F)$ is the
set $L(A) = \{w \in \Sigma^* : A accepts w\}$.
\end{definition}

\begin{definition}
  We say that two automatons are equivalent if they recognize the same language.
\end{definition}
  
\begin{definition}
  A nondeterministic finite automaton (NFA) is a quintuple
$(Q,\Sigma,\delta,q_0,F)$, where:
\begin{itemize}
  \item $Q$ is a finite set (the states);
  \item $\Sigma$ is a finite set (the alphabet);
  \item $\delta$ is a function defined over $Q x (\Sigma \cup \{\epsilon\})$ with values in
     $2^Q$ (the transition function);
  \item $q_0$ is an element of $Q$ (the initial state);
  \item and $F$ is a subset of $Q$ (the final states).
\end{itemize}
\end{definition}

\begin{definition}
  We say that the NFA $A = (Q,\Sigma,\delta,q_0,F)$ accepts the word $w = w_1
\ldots w_n$, $w_i \in (\Sigma \cup \{\epsilon \})$, $1 \leq i \leq n$, if there
is a sequence of states $r_0, \ldots, r_n$, $r_i \in Q$, $0 \leq i \leq n$, such
that:
\begin{itemize}
  \item $r_0 = q_0$;
  \item $r_{i+1} \in \delta(r_i, w_{i+1})$, $0 \leq i \leq n-1$;
  \item and $r_n \in F$.
\end{itemize}
\end{definition}
  
It is straightforward to see that every NFA has an equivalent DFA. If we view
nondeterminism through the perspective of parallel programming, we can imagine a
nondeterministic computation to be a computation where at each step the machine
creates threads corresponding to each of the possible next states. If we were
to simulate a NFA, we would just to keep track of the current states of active
threads, update them when a new symbol is read, and accept one of the threads if
any of the threads is in a final state when we reach the end of the word. The
idea is to create a DFA that runs this simulation algorithm.

% *TODO*: maybe talk about and define epsilon closure


\section{Solution}

\subsection{Representation}

\subsubsection{Automata}

The automata representation, as well as the conversion and the corresponding
proof of correctness can be found in the file Automata.scala.

For NFAs, we do not explicitly represent the alphabet, as it is not relevant for
the proof. In pratice, it is implicitly defined by the domain of the transition
function. By the same reason, we do not restrict the domain of the transition
function, but only its range. $\epsilon$-transitions are represented by
explicitly passing the value None() to the transition function. The set of final
states is substituted by a predicate that decides final states.

$\epsilon$-closure was the trickiest part of the project, and what ended up
motivating the need to implement our own representation of sets. Without the
need to prove termination, the proof would go just as well if we used lists
instead.

It is also regarding $\epsilon$-closure that one of our two assumptions
is about. We claim (and is easibly verifiable by a human by inspection) that
$\epsilon$-closure is idempotent. However, \textit{Stainless} can not verify this.

The DFA is much simpler than the NFA. We avoid representing the set of valid
states and the alphabet of work explicitly, as they are not relevant for the
proof. In theory, this structure would allow to represent non-finite
deterministic state automata. The transition function is the same as with the
NFA, except that we do not allow for $\epsilon$-transitions.


\subsubsection{Sets}

The current implementation for sets allows for traversing and stores each
element only once, by guaranteeing that the predicate \textit{contains} is
mantained as an invariant.

Other implementations were tried first. One was a wrapper around a \textit{Stainless}
list that mantained the invariant that a given list $l$ was always equal
$l.unique$, essentialy guaranteeing that each element is only stored once. This
approach, while maybe obvious, lead to proof obligations that were difficult to
prove. Another was using the \textit{Stainless} implementation for maps, but maps in
\textit{Stainless} suffer from the same shortcoming as sets, where one is not able to
traverse its keys.

The implementation is divided in two files: USet.scala and Set.scala. The former
one is where the actual implementation and proofs are done. The latter is just a
convenience wrapper that allows us to use sets without the need to explicitly
require the set invariants as preconditions everywhere we use them.

% *TODO*: subsequences

Yet another approach was to eliminate sets altogether and try to work everything
out with lists, with subsequences taking the place of subsets. However, due to
problems regarding \textit{Stainless} not supporting

\subsubsection{Memoization}

In the end, the DFA created effectively simulates the NFA and is not practical
because at every transition we need to compute the $\epsilon$-closure of some
state in the original NFA. Because of this, we created a wrapper over the DFA
representation that memoizes the transition function at every call. The new DFA
also comes with a proof of equivalence to the original DFA.

The wrapper can be found in the file AutomatonCache.scala.

\subsection{Proof}

To prove equivalence we need to prove that the NFA and the generated DFA
recognize the same words. We proved a stronger version of this. We proved that
the state of the DFA after reading some string starting from some
$\epsilon$-closed state $A$ is exactly the set of states the NFA could be in
after reading the same string starting from some state in $A$.

\subsection{Results and Discussion}

In the end, the proof is complete apart from two quirks. First,
\textit{Stainless} cannot prove that $\epsilon$-closure is idempotent, which is
a necessary fact for the proof. Secondly, \textit{Stainless} argues that the DFA
adt invariant may be violated in the NFA to DFA conversion. However, the DFA
class does not have any explicitly written preconditions, so we do not have any
explanations for this.

Both the conversion algorithm and the statement of idempotence of
$\epsilon$-closure were treated with "@library" annotations, so that
\textit{Stainless} does not consider them in the verification summary. They are
also obviously always terminating.

\end{document}